\documentclass[12pt]{article}

\title{\bf{Paper reading for Voronoi Diagram in The Laguerre Geometry and its Applications}}
\date{}
\usepackage{CJK}
\begin{document}

\maketitle

\begin{CJK}{UTF8}{bkai}

\centerline{\bf 問題定義}

本問題想要在 {\it Laguerre geometry} 上求出 {\it Voronoi diagram}。\\

將一個三維空間的點~$(x,y,z)$~對應到歐氏平面上是一個半徑為~$|z|$~且圓心為~$(x,y)$~的圓,
而且此圓的旋轉方向是根據~$z$~的正負值,則稱為 {\it Laguerre geometry}。\\

在二維空間的一圓~$C_i=C_i(Q_i;r_i)$~,其中圓心為~$Q_i=(x_i,y_i)$~
,半徑為~$r_i$~。~$C_i$~和~$P=(x,y)$~之間的距離~$d_L(C_i,P)$~之定義如下:

\begin{equation}
d_L^2(C_i,P)=(x-x_i)^2+(y-y_i)^2-r_i^2
\end{equation}

因此,在 {\it Laguerre geometry} 上 $n$ 個圓 $C_i=C_i(Q_i;r_i)(Q_i=(x_i,y_i))$ 的 {\it Voronoi polygons} 的定義如下:

\begin{equation}
V(C_i)=\cap_i \{P \in R^2 | d_L^2(C_i,P) \leq d_L^2(C_j,P) \}
\end{equation}

求出所有的 {\it Voronoi polygons} ,就是我們要的 {\it Voronoi diagram}。\\


\centerline{\bf 方法說明}

%devide

%conquer

%merge

\centerline{\bf 時間複雜度分析}

\end{CJK}

\end{document}
